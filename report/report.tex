% Class and style of the document.
\documentclass[12pt]{article}

% Package and configuration.

% Language and encoding packages.
\usepackage[french]{babel}
\usepackage[T1]{fontenc}

% Mathematics package.
\usepackage{amsmath}

% Figures package.
\usepackage{graphicx}
\usepackage{wrapfig}

% Charts package.
\usepackage{pgfplots}
\usepgfplotslibrary{units}

% Symbols package.
\usepackage{siunitx}

% Formatter package.
%% Page break between section.
\usepackage{titlesec}
\newcommand{\sectionbreak}{\clearpage}
%% Use clickable URLs.
\usepackage{url}

% Hypertext links.
\usepackage[hidelinks]{hyperref}

% Footer.
\usepackage{fancyhdr}
\pagestyle{fancy}
% Clear headers and footers.
\fancyhead{}
\renewcommand{\headrulewidth}{0pt}
\renewcommand{\footrulewidth}{0pt}
% Define footers.
\fancyfoot[RE,LO]{Pierre AYOUB}
\fancyfoot[RE,RO]{Océane FLAMANT}

\begin{document}

\title{Tri parallèle d'un tableau avec OpenMP}
\author{Pierre AYOUB -- Océane FLAMANT}

\maketitle

\tableofcontents

\newpage

\begin{abstract}
    L'objecif de ce projet est la création d'un programme, utilisant la
    bibliothèque de programmation parallèle OpenMP, permettant de trier une
    large base de données en un temps minimum.
\end{abstract}

\section{Développement du programme}

Lorem ipsum dolor sit amet, consectetur adipisicing elit, sed do eiusmod tempor
incididunt ut labore et dolore magna aliqua.

\subsection{Architecture}

Lorem ipsum dolor sit amet, consectetur adipisicing elit, sed do eiusmod tempor
incididunt ut labore et dolore magna aliqua.

\subsection{Difficultées rencontrées}

Lorem ipsum dolor sit amet, consectetur adipisicing elit, sed do eiusmod tempor
incididunt ut labore et dolore magna aliqua.

\section{Performances}

Lorem ipsum dolor sit amet, consectetur adipisicing elit, sed do eiusmod tempor
incididunt ut labore et dolore magna aliqua.

\subsection{Mesures}

Lorem ipsum dolor sit amet, consectetur adipisicing elit, sed do eiusmod tempor
incididunt ut labore et dolore magna aliqua.

\begin{figure}
    \begin{center}
        \begin{tikzpicture}
            \begin{semilogxaxis} [
                    width          = 12cm,
                    height         = 7cm,
                    axis lines     = left,
                    grid           = major,
                    grid style     = {dashed},
                    legend pos     = north west,
                    % legend style = {at={(0.2,-0.3)}},
                    ylabel         = Temps d'exécution,
                    y unit         = \si{\second},
                    xlabel         = Taille des données (N*K),
                    xtick          = data,
                    xticklabel style = {
                        /pgf/number format/precision = 3,
                    },
                    log ticks with fixed point,
                ]
                \addplot+[
                    smooth,
                    tension = 0.1,
                    color   = red,
                    mark options = {
                        fill = black,
                    }
                ]
                table [
                    x       = data size,
                    y       = exec time,
                    col sep = comma,
                ] {charts/runs.csv};
                \legend{Temps d'exécution}
            \end{semilogxaxis}
        \end{tikzpicture}
        \caption{Temps d'exécution en fonction de la taille des données (N*K)}
        \label{time-to-data-size}
    \end{center}
\end{figure}

\subsection{Perspectives d'amélioration}

Lorem ipsum dolor sit amet, consectetur adipisicing elit, sed do eiusmod tempor
incididunt ut labore et dolore magna aliqua.

\section{Conclusion}

Lorem ipsum dolor sit amet, consectetur adipisicing elit, sed do eiusmod tempor
incididunt ut labore et dolore magna aliqua.

\end{document}
