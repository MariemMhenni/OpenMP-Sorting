% Class and style of the document.
\documentclass[12pt]{article}

% Package and configuration.

% Language and encoding packages.
\usepackage[french]{babel}
\usepackage[utf8]{inputenc}
\usepackage[T1]{fontenc}

% Mathematics package.
\usepackage{amsmath}
\usepackage{amstext}

% Figures package.
\usepackage{graphicx}
\usepackage{wrapfig}

% Charts package.
\usepackage{pgfplots}
\pgfplotsset{compat=1.16}
\usepgfplotslibrary{units}
% Define style for charts.
\pgfplotsset{custom plot/.style={
        width            = 13cm,
        height           = 9cm,
        axis lines       = left,
        grid             = major,
        grid style       = {dashed},
        legend pos       = north west,
        ylabel           = Temps d'exécution,
        y unit           = \si{\second},
        enlarge y limits = upper,
        xlabel           = Taille des données,
        x unit           = N \text{ blocs}*K \text{ réels double précision},
        xticklabel style = {
            /pgf/number format/precision = 3,
        },
        log ticks with fixed point,
    }
}
\pgfplotsset{custom lines/.style={
        smooth,
        tension = 0.1,
    }
}
\pgfplotsset{custom line/.style={
        custom lines,
        color   = red,
        mark options = {
            fill = black,
        }
    }
}

% Symbols package.
\usepackage{siunitx}

% Formatter package.
%% Page break between section.
\usepackage{titlesec}
\newcommand{\sectionbreak}{\clearpage}
%% Use clickable URLs.
\usepackage{url}

% Hypertext links.
\usepackage[hidelinks]{hyperref}

% Footer.
\usepackage{lastpage}
\usepackage{fancyhdr}
\pagestyle{fancy}
% Clear headers and footers.
\fancyhead{}
% \renewcommand{\headrulewidth}{0pt}
% \renewcommand{\footrulewidth}{0pt}
% Define footers.
\fancyfoot[RE,LO]{Pierre AYOUB}
\fancyfoot[RE,CO]{\text{\thepage} sur \pageref{LastPage}}
\fancyfoot[RE,RO]{Océane FLAMANT}

% Paragraph.
\setlength{\parskip}{1ex}

\begin{document}

\title{Tri parallèle d'un tableau avec OpenMP}
\author{Pierre AYOUB -- Océane FLAMANT}

\maketitle

\begin{figure}[b]
    \centering
    \includegraphics[scale=0.3]{pictures/isty.jpg}
\end{figure}

\tableofcontents

\section{Introduction}

Durant la réalisation de ce projet, nous avions comme objectifs, dans un premier
temps, d'implémenter en langage C un algorithme de tri parallèle destiné à trier
une grande base de données. Dans un second temps, il nous fallait effectuer des
mesures du temps d'exécution de notre programme en fonction de plusieurs
paramètres, tels que le nombre de threads utilisés ou encore la taille des données.

Plusieurs contraintes nous étaient imposées :
\begin{itemize}  
    \item{Le squelette de l'algorithme à suivre était donné dans l'énoncé ;}
    \item{Nous avions à utiliser la bibliothèque de programmation parallèle OpenMP pour paralléliser notre algorithme;}
    \item{Nous avions trois fonctions principales à implémenter de la manière suivante :}
        \begin{description}
            \item[generator() :] {remplit un tableau de taille $K$ avec des nombres réels aléatoires;}
            \item[tri() :] {tri un tableau de taille $K$ ;}
            \item[tri-merge() :] {à partir de deux tableaux chacun de taille $K$ et trié, renvoie deux tableaux triés vérifiant que toutes les valeurs du premier tableau sont inférieures à celles du deuxième.}
        \end{description}
\end{itemize}

\section{Développement du programme}

Dans cette partie, nous allons expliquer et justifier les choix que nous avons
fait afin d'implémenter l'algorithme demandé, ainsi que les modifications qui
ont dû être effectuées.

\subsection{Structure du programme}

Nos fonctions \emph{generator()} et \emph{tri()} suivent scrupuleusement
l'algorithme défini dans l'énoncé. Cependant, nous avions le libre arbitre
concernant l'implémentation de la fonction \emph{tri()}.
    
Pour l'algorithme de tri, nous avons choisi le \emph{tri à bulles} (connu aussi
sous le nom de \emph{tri par permutation} ou encore \emph{tri par propagation}).
Nous reviendrons sur ces performances plus tard, mais nous pouvons souligner le
fait que c'est un tri simple à implémenter et qu'il fait partie de la famille
des tris en place : il n'a pas besoin de recopier un tableau annexe pour que le
tri s'effectue. 

\subsection{Adaptation de l'algorithme}

Nous avons rencontré quelques déconvenues avec l'algorithme donné. En effet, une
fois nos fonctions réalisées et vérifiées, nous avons lancé le programme et le
résultat obtenu n'était pas le bon. Les valeurs contenues dans les tableaux
étaient triés mais les tableaux n'étais pas dans l'odre attendu. Nous avons donc
dû effectuer une légère adaptation : en effet, en C, les indices commencent à $0$
et non à $1$, comme donné dans l'algorithme.

Tout d'abord, chaque \emph{boucle for} commence à $0$ et s'arrête à $N$. Ensuite
nous avons supprimé les $+1$ des variables $k$, $b1$ et $b2$ car les boucles
commencent à 0. Enfin, pour $b2$, le $K$ est devenu $k$ (ce dernier étant une
correction d'une erreur de l'énoncé). 

En résumé :
\begin{itemize}
    \item $k = 1+(j\bmod 2) \Rightarrow k = (j\bmod 2)$
    \item $b_1 = 1+(k+2*i)\bmod N \Rightarrow b_1 = (k+2*i)\bmod N$
    \item $b_2 = 1+(K+2*i+1)\bmodN \Rightarrow b_2 = (k+2*i+1)\bmod N$
\end{itemize}

\section{Performances}

Lorem ipsum dolor sit amet, consectetur adipisicing elit, sed do eiusmod tempor
incididunt ut labore et dolore magna aliqua.

\subsection{Mesures}

Lorem ipsum dolor sit amet, consectetur adipisicing elit, sed do eiusmod tempor
incididunt ut labore et dolore magna aliqua.

\begin{figure}
    \begin{center}
        \begin{tikzpicture}
            \begin{loglogaxis} [
                    custom plot,
                    title = $N \text{ = } \exp(\log_{10}(N*K))$ ; $K \text{ = } \frac{(N*K)}{N}$ ; $\text{OMP\_NUM\_THREADS = } \mathopen|\text{CPU cores}\mathclose|$,
                ]
                \addplot+[custom line] table [
                    x       = data size,
                    y       = exec time,
                    col sep = comma,
                ] {charts/chart1.csv};
            \end{loglogaxis}
        \end{tikzpicture}
        \caption{Temps d'exécution en fonction de la taille des données}
        \label{time-to-data-size}
    \end{center}
\end{figure}

\begin{figure}
    \begin{center}
        \begin{tikzpicture}
            \begin{loglogaxis} [
                    custom plot,
                    title = $\text{OMP\_NUM\_THREADS = } \mathopen|\text{CPU cores}\mathclose|$ ; $K \text{ = } \frac{(N*K)}{N}$,
                ]
                \foreach \i in {2,16,64,256,1024}
                    \addplot+[custom lines] table [
                        x       = data size,
                        y       = exec time n eq \i,
                        col sep = comma,
                    ] {charts/chart2.csv};
                \addlegendentry{$N = 2$};
                \addlegendentry{$N = 16$};
                \addlegendentry{$N = 64$};
                \addlegendentry{$N = 256$};
                \addlegendentry{$N = 1024$};
            \end{loglogaxis}
        \end{tikzpicture}
        \caption{Temps d'exécution en fonction de la taille des données et du nombre de blocs}
        \label{time-to-data-size-and-blocks}
    \end{center}
\end{figure}

\begin{figure}
    \begin{center}
        \begin{tikzpicture}
            \begin{loglogaxis} [
                    custom plot,
                    legend pos = north west,
                    title = $N \text{ = } \exp(\log_{10}(N*K))$ ; $K \text{ = } \frac{(N*K)}{N}$,
                ]
                \foreach \i in {1,4,16,64}
                    \addplot+[custom lines] table [
                        x       = data size,
                        y       = exec time omp_num_threads eq \i,
                        col sep = comma,
                    ] {charts/chart3.csv};
                \addlegendentry{$\text{OMP\_NUM\_THREADS} = 1$};
                \addlegendentry{$\text{OMP\_NUM\_THREADS} = 4$};
                \addlegendentry{$\text{OMP\_NUM\_THREADS} = 16$};
                \addlegendentry{$\text{OMP\_NUM\_THREADS} = 64$};
            \end{loglogaxis}
        \end{tikzpicture}
        \caption{Temps d'exécution en fonction de la taille des données et du nombre de threads}
        \label{time-to-data-size-and-threads}
    \end{center}
\end{figure}

\subsection{Perspectives d'amélioration}
    
Lorem ipsum dolor sit amet, consectetur adipisicing elit, sed do eiusmod tempor
incididunt ut labore et dolore magna aliqua.

\section{Conclusion}

Lorem ipsum dolor sit amet, consectetur adipisicing elit, sed do eiusmod tempor
incididunt ut labore et dolore magna aliqua.

\end{document}
