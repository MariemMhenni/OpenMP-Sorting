% Class and style of the document.
\documentclass[12pt]{article}

% Package and configuration.

% Language and encoding packages.
\usepackage[french]{babel}
\usepackage[T1]{fontenc}

% Mathematics package.
\usepackage{amsmath}
\usepackage{amstext}

% Figures package.
\usepackage{graphicx}
\usepackage{wrapfig}

% Charts package.
\usepackage{pgfplots}
\pgfplotsset{compat=1.16}
\usepgfplotslibrary{units}
% Define style for charts.
\pgfplotsset{custom plot/.style={
        width            = 12cm,
        height           = 9cm,
        axis lines       = left,
        grid             = major,
        grid style       = {dashed},
        legend pos       = north west,
        ylabel           = Temps d'exécution,
        y unit           = \si{\second},
        enlarge y limits = upper,
        xlabel           = Taille des données,
        x unit           = N \text{ blocs}*K \text{ réels double précision},
        xticklabel style = {
            /pgf/number format/precision = 3,
        },
        log ticks with fixed point,
    }
}
\pgfplotsset{custom lines/.style={
        smooth,
        tension = 0.1,
    }
}
\pgfplotsset{custom line/.style={
        custom lines,
        color   = red,
        mark options = {
            fill = black,
        }
    }
}

% Symbols package.
\usepackage{siunitx}

% Formatter package.
%% Page break between section.
\usepackage{titlesec}
\newcommand{\sectionbreak}{\clearpage}
%% Use clickable URLs.
\usepackage{url}

% Hypertext links.
\usepackage[hidelinks]{hyperref}

% Footer.
\usepackage{fancyhdr}
\pagestyle{fancy}
% Clear headers and footers.
\fancyhead{}
\renewcommand{\headrulewidth}{0pt}
\renewcommand{\footrulewidth}{0pt}
% Define footers.
\fancyfoot[RE,LO]{Pierre AYOUB}
\fancyfoot[RE,RO]{Océane FLAMANT}

\begin{document}

\title{Tri parallèle d'un tableau avec OpenMP}
\author{Pierre AYOUB -- Océane FLAMANT}

\maketitle

\tableofcontents

\newpage

\begin{abstract}
    L'objecif de ce projet est la création d'un programme, utilisant la
    bibliothèque de programmation parallèle OpenMP, permettant de trier une
    large base de données en un temps minimum.
\end{abstract}

\section{Développement du programme}

Lorem ipsum dolor sit amet, consectetur adipisicing elit, sed do eiusmod tempor
incididunt ut labore et dolore magna aliqua.

\subsection{Architecture}

Lorem ipsum dolor sit amet, consectetur adipisicing elit, sed do eiusmod tempor
incididunt ut labore et dolore magna aliqua.

\subsection{Difficultées rencontrées}

Lorem ipsum dolor sit amet, consectetur adipisicing elit, sed do eiusmod tempor
incididunt ut labore et dolore magna aliqua.

\section{Performances}

Lorem ipsum dolor sit amet, consectetur adipisicing elit, sed do eiusmod tempor
incididunt ut labore et dolore magna aliqua.

\subsection{Mesures}

Lorem ipsum dolor sit amet, consectetur adipisicing elit, sed do eiusmod tempor
incididunt ut labore et dolore magna aliqua.

\begin{figure}
    \begin{center}
        \begin{tikzpicture}
            \begin{loglogaxis} [
                    custom plot,
                    title = $N \text{ = } \frac{e(l(N*K)}{l(10))}$ ; $K \text{ = } \frac{(N*K)}{N}$ ; $\text{OMP\_NUM\_THREADS = } \mathopen|\text{CPU cores}\mathclose|$,
                ]
                \addplot+[custom line] table [
                    x       = data size,
                    y       = exec time,
                    col sep = comma,
                ] {charts/chart1.csv};
            \end{loglogaxis}
        \end{tikzpicture}
        \caption{Temps d'exécution en fonction de la taille des données}
        \label{time-to-data-size}
    \end{center}
\end{figure}

\begin{figure}
    \begin{center}
        \begin{tikzpicture}
            \begin{loglogaxis} [
                    custom plot,
                    title = $\text{OMP\_NUM\_THREADS = } \mathopen|\text{CPU cores}\mathclose|$ ; $K \text{ = } \frac{(N*K)}{N}$,
                ]
                \foreach \i in {2,16,64,256,1024}
                    \addplot+[custom lines] table [
                        x       = data size,
                        y       = exec time n eq \i,
                        col sep = comma,
                    ] {charts/chart2.csv};
                \addlegendentry{$N = 2$};
                \addlegendentry{$N = 16$};
                \addlegendentry{$N = 64$};
                \addlegendentry{$N = 256$};
                \addlegendentry{$N = 1024$};
            \end{loglogaxis}
        \end{tikzpicture}
        \caption{Temps d'exécution en fonction de la taille des données et du nombre de blocs}
        \label{time-to-data-size-and-blocks}
    \end{center}
\end{figure}

\begin{figure}
    \begin{center}
        \begin{tikzpicture}
            \begin{loglogaxis} [
                    custom plot,
                    legend pos = north west,
                    title = $N \text{ = } \frac{e(l(N*K)}{l(10))}$ ; $K \text{ = } \frac{(N*K)}{N}$,
                ]
                \foreach \i in {1,4,8,16}
                    \addplot+[custom lines] table [
                        x       = data size,
                        y       = exec time omp_num_threads eq \i,
                        col sep = comma,
                    ] {charts/chart3.csv};
                \addlegendentry{$\text{OMP\_NUM\_THREADS} = 1$};
                \addlegendentry{$\text{OMP\_NUM\_THREADS} = 4$};
                \addlegendentry{$\text{OMP\_NUM\_THREADS} = 8$};
                \addlegendentry{$\text{OMP\_NUM\_THREADS} = 16$};
            \end{loglogaxis}
        \end{tikzpicture}
        \caption{Temps d'exécution en fonction de la taille des données et du nombre de threads}
        \label{time-to-data-size-and-threads}
    \end{center}
\end{figure}

\subsection{Perspectives d'amélioration}
    
Lorem ipsum dolor sit amet, consectetur adipisicing elit, sed do eiusmod tempor
incididunt ut labore et dolore magna aliqua.

\section{Conclusion}

Lorem ipsum dolor sit amet, consectetur adipisicing elit, sed do eiusmod tempor
incididunt ut labore et dolore magna aliqua.

\end{document}
